\documentclass[17pt]{extarticle}
%\usepackage[paperheight=4in]{geometry}
\usepackage[top=1cm, bottom=1cm, left=2cm, right=2cm]{geometry}
\pagestyle{empty} %no page numbering
\usepackage[utf8]{inputenc}
\usepackage{graphicx}
\usepackage{amsmath}
\usepackage{amssymb}
\usepackage{amsthm}

\newtheorem{theorem}{Theorem}
\newtheorem{proposition}[theorem]{Proposition}
\newtheorem{lemma}[theorem]{Lemma}
\newtheorem{example}{Example}
\newtheorem*{example*}{Example}
\newtheorem{definition}{Definition}
\newtheorem*{definition*}{Definition}
\newtheorem{remark}[theorem]{Remark}
\newtheorem*{theorem*}{Theorem}
\newtheorem*{condition*}{Condition}

\setlength\parindent{0pt} %no indent

\begin{document}
	Let us define two families of $10$-tuples of non-negative natural numbers smaller or equal than $k$:\\
$$I_1:=\{(p_0, \cdots, p_9)\in \{0,\cdots, k\}^{10} \ : \ p_0 + \cdots + p_9 = n \}$$
$$I_2:=\{(p_0, \cdots, p_9)\in \{0,\cdots, k\}^{10} \ : \ p_0 + \cdots + p_9 = n \ \text{and} \ p_0< k\}$$
\\
With this the problem can be solved by the following formula:
$$	\sum_{(p_0, \cdots, p_9)\in I_1}
		\binom{n}{p_0,p_1, \cdots, p_9} 
		- 	\sum_{(p_0, \cdots, p_9)\in I_2}
		\binom{n-1}{p_0, p_1, \cdots, p_9} 
$$
\\ \\
\textbf{Explanation}:\\
A multinomial coefficient of the form
$$\binom{n}{p_0,p_1, \cdots, p_9} $$
presents the number of different ways $n$ elements can be colored with $9$ different colors
where a color $i$ is used exactly $p_i$ times.\\
In our example colors are presented by the digits from $0$ to $9$ and each coloring represents a 
$n$-digit number for which the first color is not $0$.\\
The first summand accounts for all colorings and the second for the colorings that start with $0$.
Thus, we obtain the number of all $n$-digit numbers in request by just subtracting the latter summand from the former.\\ \\
Multinomial coefficients can be calculated by the following formula:
$$\binom{n}{p_1,p_2, \cdots, p_m} =\frac{n!}{p_1! p_2! \cdots p_m!}$$\\

That this indeed presents the number of ways to color $n$ elements with $m$ colors by using a color
$i$ exactly $p_i$ times, can be seen as follows:\\
We construct recursively $n$ trees with $n$ levels by first taking each of the $n$ elements as root.
Further, for each such tree we add $n-1$ child nodes, by using all of the $n$ elements except the one 
being the root for the corresponding tree. We repeat this recursively for each new node, by always adding
child nodes that consist of all the elements in the current node's level except the element of the current node.
So the child nodes of the node $j_s$ with path to root 
$$j_s \rightarrow  j_{s-1} \rightarrow \cdots \rightarrow j_1$$
would be
$$\{1, 2, \cdots, n \} - \{j_1, j_2, \cdots, j_s\}$$
Since $p_1+p_2+\cdots +p_m=n$, each path from a root to a leaf can be interpreted as picking
first $p_1$ elements, then $p_2$ elements, and so on. This way, two each coloring of the $n$ elements
corresponds such a path. But this correspondence is not unique! For instance, a specific selection of
$p_1$ elements for the color $1$, has exactly as many paths corresponding to this as there are permutations
of $p_1$ elements. Note, for each permutation of the $p_1$ elements with color $1$ we may find a corresponding
path that starts with this elements. The number of these permutations exactly is $p_1!$.
Using the same arguments but applied on the $p_2$ elements colored with $2$,
we have for each permutation of the $p_1$ elements further $p_2!$ permutations of the $p_2$ elements.
Thus exactly $p_1!\cdot p_2!$ paths exist that represents the fixed coloring of $p_1$ and $p_2$.
Following up this way, we find exactly $p_1!\cdot p_2!\cdots p_m!$ pathes the present a fixed coloring of
$p_1, p_2, \cdots, p_m$.\\
The number of pathes in the tree is easily seen to be $n!$. Altogether, this gives
$$
\binom{n}{p_1,p_2, \cdots, p_m}\cdot p_1! \cdot p_2! \cdots p_m! = n!
$$

\end{document}

	
